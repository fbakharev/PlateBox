\section{Постановка задачи}

\subsection{Описание области}
Рассмотрим полосу $\Pi_0$ единичной ширины
\beq
\label{A1}
\Pi_0=\left\{x=(x_1,x_2)=(y,z)\in\bbR^2\colon x_1\in\bbR, x_2\in \left(-\frac{1}{2},\frac{1}{2}\right)\right\}
\eeq
и ее возмущение 
\beq
\label{A2}
\Pi_\ve=\left\{x\colon -\frac{1}{2}-\ve H_-(y)<z<\frac{1}{2}+\ve H_+(y)\right\},
\eeq
где $H_\pm$ --- функции с компактным носителем: $\supp H_\pm \subset [-\ell,\ell]$, $\ell>0$. Верхний и нижний ``берега'' полосы $\Pi_\ve$ обозначаем $\Gamma_\ve^\pm$. В дальнейшем нам будет удобно считать, что выполнено соотношение
\beq
\label{A2.1}
\int_{-\ell}^\ell (H_-(y)+H_+(y))y\,dy=0,
\eeq
этого можно добиться правильным выбором системы координат и величины $\ell$.

\subsection{Постановка задачи}
В области $\Pi_\ve$ рассмотрим спектральную задачу
\beq
\label{A3}
\Delta^2 u^\ve =\lambda^\ve u^\ve \quad \mbox{в} \quad \Pi_\ve, 
\eeq
для бигармонического оператора $\Delta^2$, дополненную одним из двух краевых условий: условием Дирихле
\beq
\label{A4}
u^\ve=\df_n u^\ve=0 \quad \mbox{на} \quad \df \Pi_\ve,
\eeq
или смешанным краевым условием
\beq
\label{A5}
u^\ve=\Delta u^\ve+\kp\df_n u^\ve=0 \quad \mbox{на} \quad \df \Pi_\ve,
\eeq
где $\Delta=\df_y^2+\df_z^2$ --- оператор Лапласа, $\df_n$ --- производная вдоль внешней нормали к границе полосы $\df \Pi_\ve$, а $\kp$ --- кривизна границы. Предполагается, что функции $H_\pm$ таковы, то граница $\partial \Pi_\ve$ кусочно принадлежит классу $C^2$.

Задача \eqref{A3}, \eqref{A4} в вариационной постановке принимает вид
\beq
\label{A6}
a(u^\ve,v;\Pi_\ve)=\lambda^\ve (u^\ve, v)_{\Pi_\ve} \quad \forall v\in H^2_{00}(\Pi_\ve),
\eeq
где $a(\cdot,\cdot;\Omega)$ --- билинейная форма, определенная равенством
$$
a(u,v;\Omega)= (\df_y^2u,\df_y^2v)_{\Omega}+2(\df_y\df_z u,\df_y\df_z v)_{\Omega}+ (\df_z^2u,\df_z^2v)_{\Omega},
$$
а через $(\cdot,\cdot)_\Omega$ обозначено стандартное скалярное произведение в пространстве Лебега $L_2(\Omega)$. Пространство $H^2_{00}(\Omega)$, фигурирующее в формуле \eqref{A6}, состоит из функций пространства Соболева $H^2(\Omega)$, имеющих вместе со своим градиентом нулевой след на границе $\df\Omega$.

Вариационная постановка задачи \eqref{A3}, \eqref{A5} также выражается интегральным тождеством \eqref{A6} с той лишь разницей, что функции $v$ пробегают пространство $H^2_0(\Pi_\ve)$ функций из пространства $H^2(\Pi_\ve)$ с нулевым следом на границе.

В обоих случаях билинейная форма $a$ определяет положительный самосопряженный оператор $\cA^\ve$ в пространстве $L_2(\Pi_\ve)$, который мы снабдим индексом $D$ в первом случае, и индексом $M$ --- во втором. В обоих случаях непрерывный спектр оператора является лучом $[\lambda_\dagger^D,+\infty)$
и $[\lambda_\dagger^M,+\infty)$ соответственно, а дискретный спектр может состоять из нескольких собственных чисел, расположенных ниже точки отсечки.

Точка отсечки $\lambda_\dagger$ непрерывного спектра совпадает с первым собственным числом модельной задачи на поперечном сечении, которая выражается дифференциальным уравнением 
$$
\df_z^4 U(z)= M U(z), \quad z\in \left(-\frac{1}{2},\frac{1}{2}\right)
$$
и граничными условиями
$$
U(\pm 1/2)=\df_z U(\pm 1/2)=0
$$
в случае задачи \eqref{A3}, \eqref{A4}, и
и граничными условиями
$$
U(\pm 1/2)=\df_z ^2U(\pm 1/2)=0
$$
в случае задачи \eqref{A3}, \eqref{A5}. Таким образом величины $\lambda_\dagger^D$ и $\lambda_\dagger^M$ являются корнями трансцендентных уравнений... Соответствующие собственные функции $U_\dagger^D$ и $U_\dagger^M$ выражаются формулами... 



