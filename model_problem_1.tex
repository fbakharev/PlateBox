\subsection{Модельное решение в полуплоскости с уступом}

В области 
\beq
\label{D1}
\Xi_+=\{\xi\colon \xi_1<0, \xi_2>0 \}\cup\{\xi\colon \xi_2>1\}
\eeq
рассмотрим решение краевой задачи
\begin{eqnarray}
\label{D2}
&&\Delta^2 Z(\xi)=0, \quad \xi\in \Xi_+,\\
\label{D3}
&&Z(\xi)=\df_n Z(\Xi)=0, \quad \xi\in \df\Xi_+,
\end{eqnarray}
обладающее асимптотикой
$$
Z(\xi)\sim \rho \ln\rho \sin\varphi \quad \mbox{при} \quad \rho=|\xi|\to \infty.
$$
Такое решение существует и имеет вид 
$$
Z(\xi)\sim \rho\ln \rho \sin\varphi+ K_{\Xi}\rho \sin\vph+\frac{\pi-\vph}{\pi}-K_{\Xi}\frac{\pi-\vph}{\pi} \quad \mbox{при} \quad \rho\to \infty
$$
где второе слагаемое появляется согласно общей теории, а третье и четвертое слагаемые компенсируют невязку в краевом условии.

\begin{remark}
Второе, третье и четвертое слагаемые лежат в энергетическом классе. Энергетическое пространство --- замыкание $C_0^\infty$ по норме $\|\nabla_\xi^2 u;L_2(\Xi)\|$.
\end{remark}